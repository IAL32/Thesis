\section{Premessa e ringraziamenti}

Il presente lavoro è frutto del lavoro svolto come tirocinio all'interno dell'Università di Milano-Bicocca, e viene anche utilizzato come tesi finale ai fini del conseguimento della laurea in Informatica. È però necessario chiarire che il progetto in questione non sarà abbandonato nè una volta terminata la stesura di questa relazione, nè dopo il conseguimento della laurea. È mia intenzione contribuire al meglio delle mie possibilità in quello che ritengo essere uno dei campi con il quale mi sento più legato, sia a livello di interesse professionale, che a livello strettamente personale: la ricerca sul cancro. Secondo il \textit{National Cancer Institute}, nel 2012 sono stati riportati 14.1 \textit{milioni} di nuovi casi, e di questi, 8.2 \textit{milioni} hanno portato alla morte \cite{cancerstats}. I dati mostrano anche quelli che può sembrare all'apparenza una realtà discordante: il numero totale di morti per cancro è in crescita, ma il rapporto delle morti per individuo sta calando \cite{worldindatacancer}. Nel 1990, 161 persone su 100.000 nel mondo sono morte a causa del cancro. Nel 2016, questo numero è calato a 134 su 100.000. Questo miglioramento è dovuto indubbiamente ad un numero molto elevato di fattori, tra cui l'aumento della qualità di vita ed un migliore sistema sanitario, ma è anche grazie alla crescita incessante della ricerca sul cancro, ed ai campi sui quali essa si appoggia. Lo sviluppo di algoritmi sempre più efficienti e performanti, e l'utilizzo di calcolatori super-veloci, ha permesso a questo settore di ricerca di ottenere dei considerevoli risultati.

Con questo progetto spero, quindi, di aver dato un contributo in questo settore, anche se in una percentuale minuscola.

Vorrei ringraziare mia mamma \textbf{Laura}, mio padre \textbf{José}, mia sorella \textbf{Valeria}, i miei fantastici nonni, e tutte le bellissime e meravigliose persone che hanno contribuito, in maniera diretta ed indiretta, a farmi appassionare all'informatica e, in questo caso, alla bioinformatica.