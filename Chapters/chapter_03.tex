\chapter{Inferenza di Alberi Tumorali tramite Particle Swarm Optimization}
\label{chap:pso}
In questa sezione verrà descritto lo strumento di inferenza di alberi tumorali tramite Particle Swarm Optimization (d'ora in poi \textbf{PSO}), le ipotesi effettuate, i risultati ottenuti ed eventuali conclusioni. Viene utilizzata la tecnica euristica Particle Swarm Optimization, descritta precedentemente nel \autoref{chap:intro-optim-pso}.

\section{Strumenti Utilizzati}
Lo strumento è stato sviluppato in Python. Inizialmente si era cercato di utilizzare Cython\footnote{Un linguaggio di programmazione simile a Python, ma con il vantaggio di avere velocità simili a quelle ottenute da C}, poi abbandonato poiché lo studio di questo linguaggio di programmazione avrebbe ridotto il tempo utile alla vera e propria implementazione dello strumento. Si pensa però in futuro di riscrivere il codice utilizzando \textit{C++}.

L'ambiente di sviluppo scelto è \textit{Visual Studio Code}, con l'ausilio delle estensioni di debugging per Python. In generale sono sempre stati utilizzati strumenti che seguono la filosofia \textit{Open Source}, di conseguenza questa tesi ed il codice sviluppato saranno entrambi disponibili liberamente sul profilo GitHub \cite{mygithub} del sottoscritto sotto licenza GPL3 e MIT rispettivamente.

I moduli e le librerie Python che sono state utilizzate sono elencate di seguito: \begin{itemize}
  \item \pyref{random} -- modulo standard che implementa funzioni per la generazione di numeri pseudo-randomici
  \item \pyref{multiprocessing} -- modulo standard utilizzato per implementare il supporto di parallelizzazione e gestione della concorrenza
  \item \pyref{sys} -- modulo standard per accedere ad alcune variabili o funzioni di sistema
  \item \pyref{time} -- modulo standard usato per accedere a funzioni relative al tempo, come l'accesso all'ora attuale, o per la misurazione dei tempi di esecuzione
  \item \pyref{math} -- modulo standard che permette l'accesso a funzioni ottimizzate per il calcolo matematico
  \item \pyref{copy} -- modulo standard utilizzato per copiare in profondità degli oggetti, con un nuovo puntatore all'oggetto
  \item * \pypi{graphviz} -- modulo esterno per la visualizzazione ed il salvataggio su file dei grafi da formato graphviz\footnote{Strumento che permette di disegnare rappresentare dei grafi utilizzando la notazione DOT, un linguaggio che descrive la struttura di grafi generici}
  \item * \pypi{ete3} -- modulo esterno utilizzato come base di appoggio per le funzioni essenziali relative ai nodi di un albero, come la ricerca dei nodi e l'aggiunta o rimozione di essi
  \item * \pypi{networkx} -- modulo esterno sfruttato per il calcolo del \textit{maximum weight matching}
  \item * \pypi{matplotlib} -- 
\end{itemize}

\small * Moduli esterni

\footnotetext{ASD}

\section{Adattamento del problema a Particle Swarm Optimization}
\label{chap:pso-adapt}
Una delle principali sfide in questo progetto è stata di riuscire a trovare un'interpretazione valida ed efficacie adatta alla tecnica euristica scelta, appunto il Particle Swarm Optimization, il cui funzionamento ed algoritmo sono descritti dettagliatamente nella \autoref{chap:intro-optim-pso}. Possiamo quindi dividere l'implementazione in quattro fasi: \begin{enumerate}
  \item Inizializzazione
  \item Calcolo del movimento della particella
  \item Aggiornamento della particella
  \item Aggiornamento della particella migliore dello swarm e della soluzione migliore della particella
\end{enumerate}

\subsection{Inizializzazione}
\label{chap:pso-adapt-init}
Possiamo considerare una particella dello swarm come un albero, la cui posizione è determinata dalla configurazione dei suoi nodi. L'inizializzazione viene effettuata randomizzando la topologia di un albero binario $T$ con $m$ nodi, uno per ogni mutazione, con una funzione randomica\footnote{In realtà la funzione è del tipo pseudo-randomica, utilizzando il generatore \textit{Mersenne Twister} \cite{Matsumoto:MersenneTwister}, ma non avendo problemi dal punto di vista della sicurezza o della riproducibilità, è stato pervenuto che per i nostri scopi è più che valido e sufficiente}. In questo modo, viene ottenuta una topologia simile a quella rappresentata nella \autoref{fig:pso-adapt-init-topology}. Il metodo di generazione dell'albero binario è ripreso da quello implementato su SASC, ed è descritto nell'\autoref{algo:pso-adapt-init}

\begin{algorithm}[!h]
    $m = $ numero di mutazioni
    $r = $ radice dell'albero \\
    $tree = [1, \dots, m]$ \Comment{Vettore con tutti i nodi dell'albero}
    random.shuffle(tree) \Comment{Il vettore dei nodi viene randomizzato}
    $nodes \gets [r]$ \\
    $append\_node \gets 0$ \\
    \While{$i < m$}{
      newNode $\gets$ new Node(name $=$ mutation\_name(m), parent $=$ nodes[append\_node], mutationid $=$ tree[i]) \\
      nodes.append(newNode) \\
      $i \gets i + 1$ \\
      \If{$i < m$}{
        newNode $\gets$ new Node(name = mutation\_name(m), parent = nodes[append\_node], mutationid = tree[i]) \\
        nodes.append(newNode)
      }
      append\_node $\gets$ append\_node $+ 1$
      $i \gets i + 1$
    }
    \Return{r}
    \label{algo:pso-adapt-init}
    \caption{RandomTreeInit}
\end{algorithm}

\begin{figure}[h]
  \centering
  \begin{forest}
      for tree = {grow'=0,draw}, forked edges
      [{germline}
      [{6} 
      [{18} 
      [{28} 
      [{11}  ]
      [{8}  ] ]
      [{24} 
      [{21}  ]
      [{14}  ] ] ]
      [{10} 
      [{30} 
      [{9}  ]
      [{17}  ] ]
      [{29} 
      [{1}  ]
      [{3}  ] ] ] ]
      [{23} 
      [{27} 
      [{20} 
      [{19}  ]
      [{16}  ] ]
      [{15} 
      [{12}  ]
      [{13}  ] ] ]
      [{2} 
      [{26} 
      [{7}  ]
      [{22}  ] ]
      [{4} 
      [{25}  ]
      [{5}  ] ] ] ]]
    \end{forest}    
  \caption{Esempio di topologia iniziale con $m = 30$}
  \label{fig:pso-adapt-init-topology}
\end{figure}


\subsection{Calcolo del movimento della particella}
\label{chap:pso-adapt-calculate}
La seconda fase è il calcolo della velocità con la quale una particella si muove rispetto a $p_i$ e $g$, cioè l'albero migliore che ha avuto la particella fino ad ora, e l'albero migliore in tutto lo swarm. Come parametro di paragone è definita la likelihood logaritmica descritta nell'\autoref{eq:art-sasc-lh}.

Questo step è stato uno dei problemi principali da affrontare sin dal principio del progetto. Non è immediata, data la natura non lineare del problema, trovare un modo diretto che rappresenti la velocità e la direzione di una particella verso gli alberi migliori $p_i$ e $g$. 

\section{Modello degli errori single-cell}
\label{chap:pso-matrix}
I dati single-cell vengono rappresentati tramite un modello di sostituzione come quello illustrato nella \autoref{chap:intro-models} tramite una matrice $M = n \times m$, con $n$ cellule e $m$ mutazioni. Viene poi utilizzata una matrice $D = n \times m$ è ricavata dall'osservazione dell'albero inferito, ed è una versione imperfetta del vero genotipo della matrice $M$. Per mitigare i problemi causati dalle tecniche di WGA (\autoref{chap:intro-scs}) viene utilizzato il parametro $\alpha$ per indicare la probabilità di incontrare un falso positivo, quindi osservare un $1$ quando in realtà questo è uno $0$, e falsi negativi, cioè di avere la probabilità $\beta$ di osservare uno $0$ quando in realtà questo è un $1$:
\begin{align}
    \label{eq:art-intro-model-matrix}
    \begin{split}
      P(M_{i,j} = 0 | D_{i,j} = 0) = 1 - \alpha, \qquad
      &P(M_{i,j} = 0 | D_{i,j} = 1) = \beta, \\
      P(M_{i,j} = 1 | D_{i,j} = 0) = \alpha, \qquad
      &P(M_{i,j} = 1 | D_{i,j} = 1) = 1 - \beta
    \end{split}
\end{align}

\section{Considerazioni aggiuntive}

\subsection{Guida allo strumento nello stato attuale}

\subsection{Parallelizzazione}

\section{Risultati e conclusioni}

\section{Prospettive future}
