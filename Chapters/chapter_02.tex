%- %%%%

\chapter{Stato dell'arte}

%%%%

\section{Introduzione}

Con l'avvento delle tecnologie per il sequenziamento del DNA partendo da singole cellule (SCS), iniziano ad essere disponibili dati di alta qualità. Queste tecnologie forniscono il sequenziamento di dati da singole cellule, permettendo quindi di ricostruire l'albero filogenetico di una cellula. È però da tenere in considerazione l'alto tasso di errore associato a questo tipo di dati, innalzando di conseguenza il grado di difficoltà del processo di ricostruzione della filogenesi. 
In questo capitolo, analizzeremo le tecnologie già presenti che hanno affrontato questa sfida. 

%%%%

\section{SiFit: inferring tumor trees from single-cell sequencing data under finite-sites models, 2017 \cite{zafar2017sifit} }

\subsection{Problema e soluzione}

asdf