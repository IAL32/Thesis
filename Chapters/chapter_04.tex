\chapter{Risultati e conclusioni}
\label{chap:results}

\section{Risultati su dati simulati}
\label{chap:results-sim}
I test nel capitolo precedente sono stati effettuati su dati simulati generati dal laboratorio di ricerca AlgoLab, sede dello stage, per lo strumento SASC (\autoref{chap:art-sasc}). Di questi dati è conosciuta la filogenia di partenza, rappresentata in \autoref{fig:results-phylogeny}. Per la natura aleatoria e non deterministica della ricerca dell'euristica adottata, non è possibile garantire che una volta raggiunto un ottimo (locale o globale) questo corrisponda alla filogenia di partenza. In aggiunta, nei test che seguono in questo capitolo viene utilizzata la tecnica della parallelizzazione per ridurre i tempi di esecuzione.

\begin{figure}[!h]
    \centering
    \begin{forest}
        germline
        [germline,
        [{1, 13, 20, 21, 22, 29}
            [{3, 4, 19, 25, 30},
                [11,
                    [18,
                        [{10, 15}]
                        [{17, 27, 28}]
                        [{7, 6, 16}]
                    ]
                ]
            ]
            [{5, 6, 9, 12, 23, 26}]
            [{2, 14, 24}]
        ]]
    \end{forest}
    \caption{Filogenia generata di test}
    \label{fig:results-phylogeny}
\end{figure}

\subsection{Modello di Dollo-\textit{k} con \textit{k} = 0}
Avendo a disposizione una filogenia della quale si conosce la struttura finale, si è ritenuto utile eseguire lo strumento utilizzando il parametro $k$ del modello di Dollo-\textit{k} con valore 0, poiché l'albero di partenza non presenta back mutation. La \autoref{tab:results-table-dollo-0} dei risultati mostra un'evidente peggioramento, a pari di parametri, dell'ultimo algoritmo sviluppato.

\begin{table}[H]
    \centering
    \begin{tabular}{c | c | c | c }
        Tecnica & Tempo impiegato (s) & Likelihood Iniziale & Likelihood Finale \\ \midrule \midrule 
        Assenza velocità, hill & 2216 & -7944.950318 & -677.457346 \\
        Metrica distanza, hill & 3742 (\textcolor{red}{+68.86\%}) & -7944.950318 & -689.933905 (\textcolor{red}{-1.80\%}) \\
        Clade casuali & 4159 (\textcolor{red}{+11.14\%}) & -7944.950318 & -1121.068341 (\textcolor{red}{-39.57\%})
    \end{tabular}
    \caption{Tabella di paragone delle tecniche utilizzate, $k = 0$}
    \label{tab:results-table-dollo-0}
\end{table}

\begin{figure}[!h]
    \centering
    \scalebox{0.5}{
    \begin{forest}
        germline
        [{germline} 
        [{22} 
        [{13} 
        [{21} 
        [{20} 
        [{1} 
        [{29} 
        [{19} 
        [{3} 
        [{30} 
        [{4} 
        [{25} 
        [{11} 
        [{18} 
        [{16} 
        [{8} 
        [{7} ]]]
        [{28} 
        [{27} 
        [{17} ]]]
        [{10} 
        [{15} ]]]]]]]]]
        [{23} 
        [{12} 
        [{6} 
        [{26} 
        [{5} 
        [{9} ]]]]]]
        [{2} 
        [{24} 
        [{14} ]]]]]]]]]]
    \end{forest}
    }
    \caption{Albero inferito senza parametro della velocità, stile hill climbing, $k = 0$}
    \label{fig:results-tree-dollo-0-1}
\end{figure}

\begin{figure}[!h]
    \centering
    \scalebox{0.5}{
    \begin{forest}
        germline
        [{germline} 
        [{13} 
        [{21} 
        [{22} 
        [{20} 
        [{1} 
        [{29} 
        [{3} 
        [{19} 
        [{30} 
        [{4} 
        [{25} 
        [{11} 
        [{18} 
        [{16} 
        [{8} 
        [{7} ]]]
        [{10} 
        [{15} ]]
        [{28} 
        [{27} 
        [{17} ]]]]]]]]]]
        [{14} 
        [{2} 
        [{24} ]]
        [{23} 
        [{26} 
        [{12} 
        [{6} 
        [{5} 
        [{9} ]]]]]]]]]]]]]]
    \end{forest}
    }
    \caption{Albero inferito con la metrica della distanza, stile hill climbing, $k = 0$}
    \label{fig:results-tree-dollo-0-2}
\end{figure}

\begin{figure}[!h]
    \centering
    \scalebox{0.5}{
    \begin{forest}
        germline
        [{germline} 
        [{22} 
        [{21} 
        [{3} 
        [{29} 
        [{25} 
        [{11} 
        [{4} 
        [{19} 
        [{30} 
        [{1} 
        [{13} 
        [{20} 
        [{6} 
        [{23} 
        [{12} 
        [{5} 
        [{26} 
        [{9} ]]]]]]
        [{18} 
        [{7} 
        [{16} 
        [{8} ]]]
        [{27} 
        [{28} 
        [{17} ]]
        [{14} 
        [{24} 
        [{2} ]]]]
        [{10} 
        [{15} ]]]]]]]]]]]]]]]]
    \end{forest}
    }
    \caption{Albero inferito con la metrica della distanza e clade casuali, $k = 0$}
    \label{fig:results-tree-dollo-0-3}
\end{figure}

\subsection{Modello di Dollo-\textit{k} con \textit{k} = 3}

Nella \autoref{tab:results-table-dollo-3} sono messe a paragone le tecniche adottate nel capitolo precedente, quindi utilizzando come parametri di esecuzione dell'algoritmo: $\alpha = 0.25, \beta = 1\cdot 10^{-5}, k = 3, seed = 1,\ particelle = 500,\ iterazioni = 50$. In particolare viene testata la performance sotto le condizioni del modello di Dollo-\textit{k} con $k > 0$.

\begin{table}[!h]
    \centering
    \begin{tabular}{c | c | c | c }
        Tecnica & Tempo impiegato (s) & Likelihood Iniziale & Likelihood Finale \\ \midrule \midrule 
        Assenza velocità, hill & 2408 & -7944.950318 & -694.092758 \\
        Metrica distanza, hill & 3890 & -7944.950318 & -790.961096 \\
        Clade casuali & 3744 & -7944.950318 & -1123.840910
    \end{tabular}
    \caption{Tabella di paragone delle tecniche utilizzate, $k = 3$}
    \label{tab:results-table-dollo-3}
\end{table}

\begin{figure}[!h]
    \centering
    \scalebox{0.5}{
    \begin{forest}
        germline
        [{germline} 
        [{20} 
        [{21} 
        [{22} 
        [{13} 
        [{1} 
        [{29} 
        [{3} 
        [{30} 
        [{19} 
        [{4} 
        [{25} 
        [{11} 
        [{16} 
        [{8} 
        [{18} 
        [{8},color=red 
        [{16},color=red 
        [{10} 
        [{28} 
        [{27} 
        [{17} ]]]
        [{15} ]]]]
        [{7} ]]]]]]]]]
        [{5} 
        [{23} 
        [{12} 
        [{6} 
        [{3},color=red 
        [{26} 
        [{9} ]]]]]]]]
        [{24} 
        [{14} 
        [{2} ]]]]]]]]]]
    \end{forest}
    }
    \caption{Albero inferito senza parametro della velocità, stile hill climbing, $k = 3$}
    \label{fig:results-tree-1}
\end{figure}

\begin{figure}[!h]
    \centering
    \scalebox{0.5}{
    \begin{forest}
        germline
        [{germline} 
        [{22} 
        [{20} 
        [{21} 
        [{13} 
        [{1} 
        [{29} 
        [{3} 
        [{14} 
        [{19} 
        [{4} 
        [{14},color=red 
        [{30} 
        [{25} 
        [{11} 
        [{18} 
        [{24} 
        [{10} 
        [{24},color=red 
        [{15} ]]]]
        [{16} 
        [{8} 
        [{7} ]]]
        [{28} 
        [{27} 
        [{17} ]]]]]]]]]]
        [{3},color=red 
        [{2} ]]]]
        [{12} 
        [{5} 
        [{6} 
        [{23} 
        [{9} 
        [{26} ]]]]]]]]]]]]]
    \end{forest}
    }
    \caption{Albero inferito con la metrica della distanza, stile hill climbing, $k = 3$}
    \label{fig:results-tree-2}
\end{figure}

\begin{figure}[!h]
    \centering
    \scalebox{0.5}{
    \begin{forest}
        germline
        [{germline} 
        [{20} 
        [{21} 
        [{29} 
        [{18} 
        [{3} 
        [{30} 
        [{4} 
        [{11} 
        [{19} 
        [{22} 
        [{1} 
        [{13} 
        [{12} 
        [{26} 
        [{9} 
        [{5} 
        [{23} 
        [{6} ]]]]]]
        [{24} 
        [{2} 
        [{14} ]]]
        [{25} 
        [{10} 
        [{15} ]]
        [{16} 
        [{8} 
        [{7} ]]
        [{28} 
        [{17} 
        [{27} ]]]]]]]]]]]]]]]]]]
    \end{forest}
    }
    \caption{Albero inferito con la metrica della distanza e clade casuali, $k = 3$}
    \label{fig:results-tree-3}
\end{figure}