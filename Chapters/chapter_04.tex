\chapter{Risultati e conclusioni}
\section{Risultati su dati simulati}
I test nel capitolo precedente sono stati effettuati su dati simulati generati dal laboratorio di ricerca AlgoLab, sede dello stage, per lo strumento SASC (\autoref{chap:art-sasc}). Di questi dati è conosciuta la filogenia di partenza, rappresentata in \autoref{fig:results-phylogeny}. Per la natura aleatoria e non deterministica della ricerca dell'euristica adottata, non è possibile garantire che una volta raggiunto un ottimo (locale o globale) questo corrisponda alla filogenia di partenza.

\begin{figure}[!h]
    \centering
    \begin{forest}
        germline
        [germline,
        [{1, 13, 20, 21, 22, 29}
            [{3, 4, 19, 25, 30},
                [11,
                    [18,
                        [{10, 15}]
                        [{17, 27, 28}]
                        [{7, 6, 16}]
                    ]
                ]
            ]
            [{5, 6, 9, 12, 23, 26}]
            [{2, 14, 24}]
        ]]
    \end{forest}
    \caption{Filogenia generata di test}
    \label{fig:results-phylogeny}
\end{figure}

Nella \autoref{fig:results-table} messe a paragone le tecniche adottate nel capitolo precedente.

\begin{figure}[!h]
    \centering
    \begin{tabular}{c | c | c }
        Tecnica & Tempo impiegato & Likelihood \\ \midrule \midrule 
        TT & T & L
    \end{tabular}
    \caption{Tabella di paragone delle tecniche utilizzate}
    \label{fig:results-table}
\end{figure}

\section{Prospettive future}
