\chapter{Prospettive future}
\label{chap:future}
Sebbene si tratti di uno strumento ancora in fase embrionale, esistono tutti i presupposti per poter portare avanti il progetto al di là del contesto di stage. Ci sono molte vie ancora da esplorare, e tecniche da sperimentare.

Una delle sfide principali allo attuale del progetto è il tempo di calcolo dovuto agli algoritmi di operazioni sull'albero inferito. L'algoritmo potrebbe ad esempio essere migliorato notevolmente introducendo dei calcoli matriciali. Gli alberi inferiti possono infatti essere pensati come delle matrici $n \times m$, le cui righe subiscono delle modifiche in teoria prevedibili. Inoltre, lo strumento potrebbe essere riscritto totalmente in un linguaggio più veloce e performante, come \textit{C++} oppure \textit{Cython}\footnote{Un linguaggio di programmazione che mira ad unire i vantaggi di Python con la performace di C}.

Potrebbero inoltre essere utilizzati algoritmi di ricerca in concomitanza con il PSO, come uno studio sul design di circuiti analogici integrati \cite{designanalogsapso} che ha utilizzato l'algoritmo del Simulated Annealing e il PSO.

Un'ulteriore passo in avanti potrebbe essere il tenere traccia delle likelihood passate e confrontarle tra i vari test, in maniera tale da poter verificare se e come delle inferenze vengono generate nello stesso modo, e quando.

Infine, sarebbe di essenziale aiuto creare un'interfaccia performante ed interattiva che permetta di integrare le varie tecniche utilizzate, e la visualizzazione in diretta delle operazioni subite dagli alberi utilizzati dall'algoritmo.