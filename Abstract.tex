\newpage
\cleardoublepage
\begingroup
\let\clearpage\endgroup
\null\vspace{\stretch{1}}
\chapter*{\centering Abstract}

Al fine di ricostruire gli alberi filogenetici tumorali, negli ultimi anni si è fatto uso del modello \textit{infinite-sites}, ipotizzando le progressioni tumorali come accumulazioni di mutazioni. Recenti studi che sfruttano la sequenziazione \textit{single-cell} mostrano, evidenziando la presenza di perdite di mutazioni, come questa assunzione non si riveli sempre vera. La presenza di strumenti che possano fare inferenze sulle filogenesi di alberi genetici con perdite di mutazioni è però limitata.

In questo lavoro viene illustrato ed analizzato un nuovo strumento di analisi per l'inferenza di progressioni tumorali tramite \textit{particle swarm optimization}.

\vspace{\stretch{2}} \null