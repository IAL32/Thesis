\section{Storia}

Era il 1869 quando venne isolato per la prima volta nella storia dell'umanità l'\textit{Acido Desossiribonucleico}, anche conosciuto come \textit{DNA}. Il pioniere di questa scoperta è Friedrich Miescher, medico e ricercatore nato in Svizzera nel 1844. Durante il processo di scoperta, Miescher aveva realizzato che nonostante  avesse proprietà simili alle proteine, la nuova sostanza -- il DNA -- non lo era. Prima di isolare le cellule dal pus presente nelle bende chirurgiche dell'ospedale in cui lavorava, Miescher fu molto attento ad assicurarsi che il materiale che stava utilizzando fosse fresco e non contaminato. Fu solo più tardi, nel 1871, che il ricercatore iniziò a lavorare sullo sperma di salmone, una specie di pesce che affluiva numerosa durante il periodo autunnale nella città di Basel.