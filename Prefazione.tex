% \newpage
% \cleardoublepage
% \begingroup
% \let\clearpage\endgroup
\null\vspace{\stretch{1}}
\chapter*{\centering Prefazione}

Il presente lavoro è stato svolto sotto la guida ed il supporto di AlgoLab, laboratorio presso il dipartimento di informatica dell'Università di Milano-Bicocca, che ha lo scopo di progettare, studiare, analizzare ed implementare algoritmi efficienti per problemi computazionali. Il tirocinio è cominciato il 22 Marzo 2019, ed è stato condotto per la maggior parte in maniera autonoma, da remoto.
Il problema affrontato è l'\textit{inferenza di progressioni tumorali} su dati single-cell, al fine di determinare l'ordine e la frequenza con cui le variazioni somatiche vengono acquisite durante una progressione tumorale. Spesso ciò è basato sulla “Infinite Sites Assumption”, dove le mutazioni possono solo essere acquisite, e mai perse. Lo stage si colloca nella ricerca del superamento di tale assunzione, utilizzando il modello della \textit{filogenesi persistente}, dove ogni mutazione può essere persa al massimo una volta nell'intero albero. Più precisamente, si è investigata la tecnica \textit{Particle Swarm Optimization}, un algoritmo di ottimizzazione di tipo euristico, ispirato al movimento degli sciami. I dati single-cell sono caratterizzati da un elevato tasso di errore e di valori mancanti: ciò rende inutilizzabili gli approcci noti in letteratura per i dati di \textit{bulk sequencing}. In particolare, sono state analizzate quali strutture dati utilizzare per rendere l'algoritmo efficiente ed efficace, e quali operazioni considerare per inferire predizioni accurate.

\vspace{\stretch{2}} \null
