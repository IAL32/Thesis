\documentclass[a4paper,12pt,openright]{book}
% Per gli accenti e caratteri speciali
\usepackage[utf8]{inputenc}
\usepackage[T1]{fontenc}
\usepackage[paper=a4paper,margin=1in]{geometry}
\usepackage{csquotes}
% Formule matematiche
\usepackage{amsmath, amssymb, amsfonts, latexsym, commath}
\usepackage[italian]{babel}
\usepackage{setspace}
\usepackage{listings}
\usepackage{xcolor}
\usepackage{emptypage}
% Inserimento di immagini
\usepackage{graphicx}
\graphicspath{ {./Images/} }
% Per le formule di biologia
\usepackage{xymtex}
\usepackage{wrapfig}
\usepackage{caption}
\usepackage{subcaption}
% \usepackage[lofdepth,lotdepth]{subfig}
% Link esterni e interni
\usepackage[colorlinks=true,urlcolor=black,linkcolor=black,citecolor=black]{hyperref}
\usepackage{float}
\usepackage{booktabs}
% Referenze
\usepackage{biblatex}
\usepackage{fancyhdr}
% Grafici carini
\usepackage{pgfplots}
\usepackage{tikz}
\usepackage[external]{forest}
% Renderizzo solo quando necessario, se già renderizzato
% viene incluso direttamente, velocizzando drasticamente
% il tempo di compilazione
\usetikzlibrary{external,calc,automata,chains,arrows.meta}
\usepgfplotslibrary{external} 
\tikzexternalize[prefix=tikzfigures/] % activate!
% Algoritmi
\usepackage[linesnumbered,ruled,vlined]{algorithm2e}

\geometry{top=21mm,bottom=21mm}
\setcounter{secnumdepth}{5}
\batchmode
\bibliography{bibliography}
\pgfplotsset{compat=newest}
\pagestyle{fancy}

\newcommand{\M}[1]{$\mathrm{#1}$}

% Adenina
\def \adenina{
	\let\substfont=\sffamily
	\let\substfontsize=\footnotesize
	%\changeunitlength{0.1pt}
	\purinev{4==NH\M{_2};6==H;2==H;1==H}
}

% Guanina
\def \guanina{
	\let\substfont=\sffamily
	\let\substfontsize=\footnotesize
	%\changeunitlength{0.1pt}
	\purinev{4D==O;5==H;6==NH\M{_2};2==H;1==H}
}

% Citosina
\def \citosina{
	\let\substfont=\sffamily
	\let\substfontsize=\footnotesize
	%\changeunitlength{0.1pt}
	\pyrimidinevi[e]{1==H;2D==O;4D==NH\M{_2};5==H;6==H}
}

% Timina
\def \timina{
	\let\substfont=\sffamily
	\let\substfontsize=\footnotesize
	%\changeunitlength{0.1pt}
	\pyrimidinevi[e]{1==H;2D==O;3==H;4D==O;5==CH\M{_3};6==H}
}

% rotates text by #4 degrees
\newcommand{\mcrot}[4]{\multicolumn{#1}{#2}{\rlap{\rotatebox{#3}{#4}~}}} 

\pagenumbering{roman}

\lhead{E3101Q - Informatica}
\chead{}
\rhead{}
\lfoot{Dipartimento di Informatica}
\cfoot{\thepage\ }
\rfoot{\small Università degli Studi di Milano-Bicocca}

\author{Adrian David Castro Tenemaya, 816015}
\title{Università degli Studi di Milano-Bicocca \\{\small Dipartimento di Informatica \\ Corso di Informatica - E3101Q\\A.A. 2016-2019\\\vspace*{0.55in} \ }\\ Inferenza di alberi tumorali tramite Particle Swarm Optimization \vspace*{1.25in}}
\date{\today}

%%%%%%%%%%%%%%%%
\begin{document}
%%%%%%%%%%%%%%%%

% Frontespizio da https://github.com/lucach/frontespizio-unimib
\begin{titlepage}
    \noindent
    \begin{minipage}[t]{0.19\textwidth}
        \vspace{-4mm}{\includegraphics[scale=1.15]{logo_unimib.pdf}}
    \end{minipage}
    \begin{minipage}[t]{0.81\textwidth}
        \setstretch{1.42}
        \textsc{Università degli Studi di Milano - Bicocca} \\
        \textbf{Scuola di Scienze} \\
        \textbf{Dipartimento di Informatica, Sistemistica e Comunicazione} \\
        \textbf{Corso di laurea in Informatica} \\
        \par
    \end{minipage}
    \vspace{40mm}
    \begin{center}
        \large \setstretch{1.2}
        \textbf{Inferenza di alberi tumorali tramite \\ Particle Swarm Optimization}
        \par
    \end{center}
    \vspace{50mm}
    \noindent
    {\large \textbf{Relatore:} Prof. Della Vedova Gianluca } \\
    \noindent
    {\large \textbf{Co-relatore:} Dott. Ciccolella Simone}
    \vspace{15mm}
    \begin{flushright}
        {\large \textbf{Relazione della prova finale di:}} \\
        \large{Castro Tenemaya Adrian David} \\
        \large{Matricola 816015}
    \end{flushright}
    \vspace{40mm}
    \begin{center}
        {\large{\bf Anno Accademico 2018-2019}}
    \end{center}
    \restoregeometry
\end{titlepage}
% \maketitle

% \makeatletter\@openrighttrue\makeatother
\newpage
\tableofcontents

% \makeatletter\@openrightfalse\makeatother


\newpage
\cleardoublepage
\begingroup
\let\clearpage\endgroup
\null\vspace{\stretch{1}}
\chapter*{\centering Premessa}

Il presente lavoro è frutto del lavoro svolto come tirocinio all'interno dell'Università di Milano-Bicocca, e viene anche utilizzato come tesi finale ai fini del conseguimento della laurea in Informatica. È però necessario chiarire che il progetto in questione non sarà abbandonato nè una volta terminata la stesura di questa relazione, nè dopo il conseguimento della laurea. È mia intenzione contribuire al meglio delle mie possibilità in quello che ritengo essere uno dei campi con il quale mi sento più legato, sia a livello di interesse professionale, che a livello strettamente personale: la ricerca sul cancro. Secondo il \textit{National Cancer Institute}, nel 2012 sono stati riportati 14.1 \textit{milioni} di nuovi casi, e di questi, 8.2 \textit{milioni} hanno portato alla morte \cite{cancerstats}. I dati mostrano anche quelli che può sembrare all'apparenza una realtà discordante: il numero totale di morti per cancro è in crescita, ma il rapporto delle morti per individuo sta calando \cite{worldindatacancer}. Nel 1990, 161 persone su 100.000 nel mondo sono morte a causa del cancro. Nel 2016, questo numero è calato a 134 su 100.000. Questo miglioramento è dovuto indubbiamente ad un numero molto elevato di fattori, tra cui l'aumento della qualità di vita ed un migliore sistema sanitario, ma è anche grazie alla crescita incessante della ricerca sul cancro, ed ai campi sui quali essa si appoggia. Lo sviluppo di algoritmi sempre più efficienti e performanti, e l'utilizzo di calcolatori super-veloci, ha permesso a questo settore di ricerca di ottenere dei considerevoli risultati.

Con questo progetto spero, quindi, di aver dato un contributo in questo settore, anche se in una percentuale minuscola.

\vspace{\stretch{2}} \null

\newpage
\cleardoublepage
\begingroup
\let\clearpage\endgroup
\null\vspace{\stretch{1}}
\chapter*{\centering Ringraziamenti}
\begin{flushright}
    \textit{
        Ai miei genitori Laura e José, per praticamente tutto; \\
        A mia sorella Valeria; \\
        Ai miei fantastici nonni; \\
        A Luca, Alessio, Lorenzo, Ilaria, Luisa, Davide, Nicoletta, Laura. Grandi amici senza i quali nulla di tutto questo sarebbe stato possibile; \\
        A chi ha rallegrato le mie serate durante le feste a casa; \\
        Alle bellissime e meravigliose persone che hanno contribuito, in maniera diretta ed indiretta, a farmi appassionare all'informatica e, in questo caso, alla bioinformatica; \\
        A chi mi è stato vicino in tempi bui; \\
        A Iris ed alla sua enorme pazienza nel sopportare le mie pessime battute.
    }
\end{flushright}

\vspace{\stretch{2}} \null

\input{Prefazione}
% \newpage
% \cleardoublepage
% \begingroup
% \let\clearpage\endgroup
\null\vspace{\stretch{1}}
\chapter*{\centering Abstract}

Al fine di ricostruire gli alberi filogenetici tumorali, negli ultimi anni si è fatto uso del modello \textit{infinite-sites}, ipotizzando le progressioni tumorali come accumulazioni di mutazioni. Recenti studi che sfruttano la sequenziazione \textit{single-cell} mostrano, evidenziando la presenza di perdite di mutazioni, come questa assunzione non si riveli sempre vera. La presenza di strumenti che possano fare inferenze sulle filogenesi di alberi genetici con perdite di mutazioni è però limitata.

In questo lavoro viene illustrato ed analizzato un nuovo strumento di analisi per l'inferenza di progressioni tumorali tramite \textit{particle swarm optimization}.

\vspace{\stretch{2}} \null
\newpage

% \makeatletter\@openrighttrue\makeatother

\foreach \i in {01,02,03,04,05} {%
    \edef\FileName{Chapters/chapter_\i}%     The % here are necessary to eliminate any
    \IfFileExists{\FileName}{%  spurious spaces that may get inserted
       \input{\FileName}%       at these points
    }
}

\listoffigures

\listoftables

\printbibliography

% In order to be able to reconstruct phylogenetic trees of tumors, in the last years the infinite-sites model has been used, assuming tumoral progressions as an accumulation of mutations. Recent studies using the single-cell sequencing technique show, highlighting the presence of mutation losses, how this assumption is not always true. However there is a lack of tools that are able to infer the phylogeny of genetic trees while allowing loss of mutations. In this work it is presented a new tool

%%%%%%%%%%%%%%
\end{document}
%%%%%%%%%%%%%%